% プロジェクト学習中間報告書書式テンプレート ver.1.0 (iso-2022-jp)

% 両面印刷する場合は `openany' を削除する
\documentclass[openany,11pt,papersize]{jsbook}

% 報告書提出用スタイルファイル
%\usepackage[final]{funpro}%最終報告書
\usepackage[middle]{funpro}%中間報告書

% 画像ファイル (EPS, EPDF, PNG) を読み込むために
\usepackage[dvipdfmx]{graphicx,color}

% ここから -->
\usepackage{calc,ifthen}
\newcounter{hoge}
\newcommand{\fake}[1]{\whiledo{\thehoge<70}{#1\stepcounter{hoge}}%
  \setcounter{hoge}{0}}
% <-- ここまで 削除してもよい

\usepackage{here}

% 年度の指定
\thisYear{2015}

% プロジェクト名
\jProjectName{フィールドから創る地域・社会のためのスウィフトなアプリ開発}

% [簡易版のプロジェクト名]{正式なプロジェクト名}
% 欧文のプロジェクト名が極端に長い(2行を超える)場合は、短い記述を
% 任意引数として渡す。
%\eProjectName[Making Delicious curry]{How to make delicious curry of Hakodate}
\eProjectName{``Swift'' Application Development Based on Field Research}


% <プロジェクト番号>-<グループ名>
\ProjectNumber{3-C}

% グループ名
\jGroupName{教育系グループ}
\eGroupName{Education Group}

% プロジェクトリーダ
\ProjectLeader{1013220}{新保遥平}{Yohei~Shinpo}

% グループリーダ
\GroupLeader  {1013015}{中進吾}{Shingo~Naka}

% メンバー数
\SumOfMembers{5}
% グループメンバ
\GroupMember  {1}{1013130}{熊谷優斗}{Yuto~Kumagai}
\GroupMember  {2}{1013116}{皀勢也}{Seiya~Kurokome}
\GroupMember  {3}{1013220}{新保遥平}{Yohei~Shinpo}
\GroupMember  {4}{1013015}{中進吾}{Shingo~Naka}
\GroupMember  {5}{1013104}{矢吹渓悟}{Keigo~Yabuki}

% 指導教員
\jadvisor{伊藤恵,奥野拓,原田泰,木塚あゆみ,南部美砂子}
% 複数人数いる場合はカンマ(,)で区切る。カンマの前後に空白は入れない。
\eadvisor{Kei~Itou,Taku~Okuno,Yasushi~Harada,Ayumi~Kizuka,Misako~Nambu}

% 論文提出日
\jdate{2015年7月29日}
\edate{July~29, 2015}

%%%%%%%%%%%%%%%%%%%%%%%%%%%%%%%%%%%%%%%
\usepackage{graphicx}
%%%%%%%%%%%%%%%%%%%%%%%%%%%%%%%%%%%%%%%
\begin{document}
%
% 表紙
\maketitle

%前付け
\frontmatter

% 和文概要
\begin{jabstract} 
%\fake{ここに日本語の概要を書きます。}
 本プロジェクトは教育というフィールドを調査し、教育に関する問題を解決するiOSアプリを開発することを目的としている。

 各メンバーが教育に関わるアプリを考え、メンバーと担当教員にプレゼンテーションを行った。メンバー間では情報の共有を行い、担当教員からはレビューを受けた。その後、担当教員から受けたレビューを基にお互いにアイデアを広げ、テーマを1つに絞り込み大学生向けプログラミング入門アプリに決まった。

 テーマが決まった後、アプリの設計を行った。しかし、要件定義を固めずにアプリの設計を行ったため、一貫性のないアプリ設計になってしまった。そのため、要件定義を1からやり直すことになった。要件定義をやり直すことは、1度で終わらず何度も行った。その結果、中学校でプログラミングを学んだ人、興味を持った人を対象にしたゲームアプリというテーマになった。

 現在日本の中学校では、2012年から中学校の技術家庭科でプログラミング教育が必修項目となっている。しかし、今の中学校のプログラミング教育ではソースコードを打ってプログラミングをするということを行っておらず、プログラミングの内容を深く取り上げていない。そこで、私たちは中学で学んだプログラミングと実際のプログラミングの間のプロセスを支援するゲームアプリを開発することを決めた。

 中間発表では、私たちが考えた提案をポスターにまとめ、ポスターセッションを行った。教員や他学生からの評価シートには「最終的なゴールは?」、「まだ内容が決まっていないので評価不能」、「既存のもとの比較がない」などの意見をいただき、もう1度要件定義を見直しアプリの設計をやり直す必要があることに気付かされた。

 今後は、アプリの設計をやり直し、実装を行うことを考えている。また、11月に開催されるアカデミックリンクにてワークショップを開き、そこで得たレビューを基にアプリを改善していくことを考えている。

% 和文キーワード
%\begin{jkeyword}
%キーワード1, キーワード2, キーワード3, キーワード4, キーワード5
%\end{jkeyword}
\bunseki{中進吾}
\end{jabstract}

%英語の概要
\begin{eabstract} 
%\fake{you should write your English abstract in one page. }
 This project is having for its object to investigate a field as education and develop the iOS application useful for education.

 Each member considered the application of educating and presented members and teachers. We shared information among the members and received reviews from teachers. After those ideas was expanded each other and the theme was narrowed down to 1 based on the review we received from teachers. The theme was decided in programming guide application for college students.

 After the theme was decided, the application was designed. But the design of application had inconsistent because the application was designed without making the requirement definition hard. Therefore we changed the requirement definition from one. We did not finish changing the requirement definition and went many times. As a result, it was decided in the theme as the game application that made the person who learned a programming at junior high school and interested people the subject.

 Programming education is the compulsory item at technical and homemaking course of Japanese junior high school from 2012. But neither to hit source cord with programming education of the junior high school and program be being performed nor the contents of programming be taken up deeply now. So we have decided to develop the game application which supports process between the programming learned at junior high school and the actual programming.

 The proposition that we thought was gathered in a poster and a poster session was performed in the middle announcement. We received opinions of which ``what is last goal'', ``having no comparison it exists down'', ``the contents are not decided yet, so evaluation is impossible'' in an evaluation seat from teachers and other students, and the requirement definition was reconsidered again, and they made notice that it is necessary to redo design of an application.

 We are thinking about design and redevelopment of our application from now on. A workshop will be opened in the Hakodate Academic Link held in November and we get the review and are thinking an application is being improved. 
% 英文キーワード
%\begin{ekeyword}
%Keyrods1, Keyword2, Keyword3, Keyword4, Keyword5
%\end{ekeyword}
\bunseki{中進吾}
\end{eabstract}

\tableofcontents% 目次


\mainmatter% 本文のはじまり

\chapter{前期の背景}
\section{世界と日本のプログラミング教育について}
現在、世界中でプログラミング教育の必要性が高まっている。政府が公教育としてプログラミングを取り入れている、または取り入れようとしている国が増えてきている。イギリスでは、5歳から16歳の義務教育の新カリキュラムにプログラミングが正式導入されており、エストニアでは小学校1年生からアプリ開発の授業を開始することになっている[1]。日本でも2012年から中学校の技術家庭科で、プログラミング教育が必修項目となっている[2]。

日本では、ビジュアルプログラミング言語の図1.1のScratchや図1.2のビュートビルダーなどを用いて、ラジコンなどの機械を動かす授業を行っている[3][4][5]。
\begin{figure}[H]
\begin{center}
\includegraphics[width=8cm, bb=0 0 1306 780]{img/Scratch.jpg}
\end{center}
\caption{Scratchの画面}
\end{figure}



\begin{figure}[H]
\begin{center}
\includegraphics[width=8cm, bb=0 0 1006 770]{img/BeautoBuilderP_SSs.png}
\end{center}
\caption{ビュートビルダーの画面}
\end{figure}
また、2013年6月5日に安倍政権の経済政策「アベノミクス」の「第3の矢」として発表した成長戦略の素案には、「産業競争力の源泉となるハイレベルな IT人材の育成・確保」という項目があり、その中には「義務教育段階からのプログラミング教育等のIT教育を推進する」との記載があった[1]。今後、日本のプログラミング教育はさらに拡大していくことが予想される。

しかし、今の中学校のプログラミング教育ではソースコードを打ってプログラミングをするということを行っておらず、プログラミングの内容を深く取り上げていない。また、プログラミングを学べるのは中学校3年生の時だけで、イギリスやエストニアと比べるととても短い期間である。高校では、実際にプログラミングを教えているところもあるが、義務化さていないので誰もが学校でプログラミングを学べるわけではない[6]。
\bunseki{中進吾}

\section{現状と課題}
日本の中学校ではビジュアルプログラミング言語を用いたプログラミングの授業を行っており、ソースコードを書く練習は行っていない。ビジュアルプログラミング言語はC言語やJavaのようなプログラミング言語と表記の仕方が大きく異なっている。そのため中学校の授業だけでは、C言語やJavaのように実際に文字を打ち込むようなソースコードを組もうとした時、どのように組んでいいかわからない。Webサイトやアプリなどのシステム開発を行う際、基本は文字を打ち込むプログラミング言語を用いるので、ビジュアルプログラミング言語はほとんど使用しない。今の中学校のプログラミング教育だけでは、産業競争力の源泉となるハイレベルな IT人材の育成・確保をすることはできない。現状のままでは、イギリスやエストニアなどの他国との差が広がる一方である。
\bunseki{中進吾}

\chapter{前期のプロジェクトの目標}
\section{開発アプリの目標}
背景で述べたように、世界中でプログラミング教育の必要性が高まっており、実際に小学生からアプリ開発の授業を行っている国もある。日本でも中学校の技術家庭科でプログラミング教育が必修項目となっている。しかし、現在の中学校のプログラミング教育ではソースコードを打ってプログラミングをするということをしておらず、プログラミングの内容を深く取り上げていない。
\par
そのため、中学校でプログラミングを学んだ人、興味をもった人を対象として、中学で学んだプログラミングと実際のプログラミングの間のプロセスを支援し、ソースコードの組み立てを学ぶことが出来るゲームアプリを開発する。そこで、ビュートビルダーやScratchのようなビジュアルプログラミング言語を学んだ中学生が、C言語のように実際に文字を打ち込むようなソースコードの組み方を理解できるようになることが、開発アプリの目標である。
\bunseki{皀勢也}

\section{プロジェクト学習としての目標}
新しいプログラミング言語であるSwift言語を使い、アジャイル開発手法の一つであるScrumという方法論を用いて素早いアプリ開発をする。また、短期間での開発とフィードバックを繰り返し、より良い品質のアプリ開発を目指している。さらに、バージョン管理システムの理念を学び、効率よくアプリ開発する。プロジェクト学習を通して、情報システムコース、高度ICTコースや情報デザインコースなど異なるコースのメンバーで開発を進めていく。開発を進めていく中で、コミュニケーション能力、グループ開発力を養い、異なる分野の知識を吸収する。
\par
プロジェクト学習としての最終的な目標はアカデミックリンク、成果発表会や課外発表会でアプリの発表を行いレビューを受け、受けたレビューを反映させたアプリをリリースすることである。
\bunseki{皀勢也}

%3章
\chapter{前期の活動}

\section{プロジェクト全体としての活動}

\subsection{スクラッチワークショップへの参加}
\par 教育をテーマにするに当たり、まず小学生と触れ合い、教育の現状について考えるために、原田先生主催のワークショップに参加した。ワークショップは、2015年5月9日に、函館市青年センターにて行われ、小学校5年生から中学校1年生の子供達計10人が参加した。ワークショップの内容は、ビジュアルプログラミング言語「Scratch」を用いて、動きに反応して音が鳴る不思議楽器を作るというものである。当日、メンバーは小学生の側についてプログラミングのアシスタントをした。
\par ワークショップを通して、気づいた点は次の2点である。
\begin{itemize}
\item 子供達は、一度得た知識はすぐに自分のものにしているようだった。今回のワークショップは、前回のワークショップ参加者から引き続き参加している子供が多いということもあって、メンバーが使い方を教えるまでもなく、自力でプログラミングを行っていた。更に、繰り返し文の使い方を教えたところ、「じゃあさ、ここもこうすればいいんじゃない?」と、子供自ら別の点の修正を行っていた。子供の成長能力の高さに驚いた。
\item 前回から参加している子供に、どうして今回も参加したの?と尋ねたところ、「だって、これ(Scratch)楽しいんだもん」と答えた。子供でもプログラミングに興味を持っていることに驚いた。
\end{itemize}

\par また、ワークショップの最後に、参加者の子供達と、その親に向けた簡単なアンケートを実施した。しかし、プロジェクトとしての方針が決まってない状態で作成したアンケートだったため、内容が建設的なものではなく、得たアンケート結果をその後に生かすことが出来なかった。むしろ、アンケート内容に子供にはわかりづらい表現がある、難しい漢字を使っている、子供用と大人用のアンケート用紙の区別がつかないといった問題を発見できたことが、その後に生かせる学びであったと言える。

\bunseki{熊谷優斗}

\subsection{リスク分析}
\par プロジェクトを進めるにあたって起こりうるリスクをメンバーそれぞれで洗い出し、それぞれのリスクに対して発生確率、被害の内容、対処方法を挙げた。図3.1は洗い出したリスクの一部である。

\par リスクの洗い出しをした時点で既に発生していたのが、「メンバーに連絡がつながらない」というリスクだ。前述のスクラッチワークショップにてアンケートを実施したが、このアンケートを作成する際、メンバーの1人に連絡がつながらず、ワークショップ当日になってそのメンバーにアンケート内容のレビューをしてもらった結果、いくつかの不備があることが発覚した。この不備は、そのメンバーが前日にアンケート内容をレビューできていれば気づけたはずである。今後このようなリスクが発生しないよう、メンバー内で1日1回はSkypeやLINEを確認することを義務づけた。

\begin{figure}[H]
\begin{center}
\includegraphics[width=14cm, bb=0 0 753 394]{img/RiskManagement.png}
\end{center}
\caption{リスク分析の結果(一部抜粋)}
\end{figure}

\bunseki{熊谷優斗}

\subsection{アプリ開発のための勉強会}
\par iOSアプリを開発するにあたって必要となる知識を学ぶ勉強会をプロジェクトのティーチングアシスタントが開催したため、これにグループ全員で参加した。勉強会では、XcodeやSwift言語の使い方を学ぶSwift勉強会とバージョン管理システムである、GitとGitHubの使い方を学ぶGitHub勉強会の2種類が行われた。それぞれで行ったことを具体的に記述する。
\par Swift勉強会は全部で3回行われた。第1回では、メンバーそれぞれのPCにXcodeを導入し、Swift言語によってIBLabelやIBButtonを用いた簡単なアプリを作成した。その後、iPadにて作成したアプリをビルドするために、iOS Developer Programへの登録を行った。第2回では、MapKitというFrameworkを用いた地図アプリを作成した。第3回では、サーバーからデータを読み書きすることのできるアプリを作成した。それぞれの回の終わりには演習問題が出され、これを解くことで学んだ知識の復習を行うことができた。
\par GitHub勉強会は全部で3回行われた。それぞれの回を通して、バージョン管理システムの理念を学びつつ、Gitの基本的な使い方を学んでいった。第3回では、Swift勉強の演習問題をGitHubを用いてメンバー間で分担しながら作成せよ、という課題が出た。しかし、上手くコーディングの役割分担を行うことができず、1人で全てのコーディングをし、残りのメンバーでコードレビューをするという形を取った。これに対し、ティーチングアシスタントから昨年度はもっと役割分担ができていた、という報告を受けた。今後、上手く役割分担をしていくために教育班では、GitHub の issue 機能を利用していくことを決定した。
\bunseki{熊谷優斗}

\subsection{バックログの作成}
\par プロジェクトの方針として、アジャイル開発手法の1つであるScrumという方法論を取り入れることが決まっていたため、プロジェクトのスケジュールをバックログを用いて管理した。バックログとは成果物を作り出すために必要な要素を項目に起こした一覧のことで、この一覧を上下に整頓することで項目の優先順位を表す。バックログには明確なスケジューリングをする必要はなく、優先順位の高いものから順番に行っていく。図3.2は6、7月分のバックログの原案である。

\par この原案を企業講師である高森満様と木下実様にお見せしたところ、「バックログの優先度を議論する際にもっと手軽に入れ替えることが可能なように、紙や付箋を用いたほうが良い」というレビューを受けた。そこで、せっかく紙と付箋を使用するならばと、ソフトウェア開発のツールの1つである、「タスクかんばん」のシステムをバックログに取り入れることにした。具体的にはタスクの状態を「TODO」、「DOING」、「DONE」の3つのステージに分割し、更に「TODO」欄のタスクの上下関係によってタスクの優先度を表すようにした。図3.3は実際に使用しているバックログである。

\begin{figure}[H]
\begin{center}
\includegraphics[width=14cm, bb=0 0 1020 359]{img/SprintBacklog.png}
\end{center}
\caption{6、7月分のバックログの原案}
\end{figure}

\begin{figure}[H]
\begin{center}
\includegraphics[width=14cm, bb=0 0 1206 783]{img/TaskKanban.png}
\end{center}
\caption{タスクかんばんのシステムを取り入れたバックログ}
\end{figure}

\bunseki{熊谷優斗}

\subsection{中間発表会の資料制作}
\par 中間発表会に向けて、ポスターを制作した。制作にあたって、グループメンバーを実装班3人とポスター班2人に分け、ポスターの制作が終わったら実装班がレビューをする、という形式をとった。しかし、メンバー間の意識共有が上手く行われていなかったため、実装班がポスターのレビューを上手く行うことが出来なかった。そのため、ポスターをティーチングアシスタントや担当教員に見せたところ、目的と制作物がずれている、というレビューを受けた。これを受けて、メンバー5人全員で、一度背景、目的、課題の見直しを行い、ポスターの作り直しをした。しかし、短期間で急いでポスターの作り直しを行ったため、今度は文字が多すぎて見づらいというレビューを受けた。これを受けて、ポスター内の文字を少なくするため、もう一度ポスターの構成を見直すという作業を行った。結果、ポスターを作り上げることができたが、その制作に多くの時間を割くことになってしまった。ポスターやその他ドキュメントを作る際には、まずメンバー間の意識共有を行い、どういった構成で文書を書いていくのかを考えることに時間をかけるべきだ、ということを学んだ。

\bunseki{熊谷優斗}

\subsection{中間発表会}
\par 中間発表会ではメンバーを前半3人、後半2人に分けて、発表を行った。前半の発表では、アプリのデモを行わないと内容が伝わりづらい、というレビューを受けた。これを受けて後半の発表では、デモを取り入れ、内容が伝わりやすいようにした。しかし、後半の発表では合計で9人しか見に来た人がいなかった。このことから、教育系が開発しているアプリに魅力が少ないのではないか、という気づきを得た。

\bunseki{熊谷優斗}

\section{アプリ案の変化と内容}
\par プロジェクトを進めるうちに、アプリ内容、対象ユーザーが変化していった。その内容を以下に記す。
\bunseki{熊谷優斗}

\subsection{アプリ案の検討}
\par メンバーそれぞれが考えて来た案を評価し、5種類の案に絞った。図3.4に案を表す。

\par 5種類それぞれの案をメンバー内で肉付けした後、ティーチングアシスタントと担当教員からレビューを受けた。それぞれの案の詳細とレビューの内容を以下に示す。

\begin{description}
 \item[案1 いじめ対策アプリ]
従来相談を受けてもらうためには、電話をかけ、言葉で喋らなければならないので、ハードルが高い。一部の教育委員会では、メールの対応も行っている。そこで、少しでもハードルを下げるために、LINEのように教育委員会と会話ができるようにする。
\begin{description}
 	\item[レビュー内容]

	このままだとただチャットをするだけのアプリになってしまうのではないか。既存のSNSアプリと差別化を行うため、独自の機能が必要である。また、どのようにしてこのアプリの評価を行うかという点は要検討である。
	 \end{description}
 
  \item[案2 プログラミングを学ぶゲームアプリ]
子供がゲーム攻略を楽しみながら、いつのまにかプログラミングを覚えることができるアプリである。最終的なユーザーの到達点としては制御文が使えるようになることである。
	\begin{description}
 	\item[レビュー内容]
	ゲーム内容は、答えを導くのに手間がかかり、ユーザーに達成感があるものにすべきである。似たようなアプリは既にいくつも存在しているため、それらを調査し、どのように差別化を図るか検討する必要がある。
	 \end{description}

  \item[案3 発想力を鍛える−生産消費的なゲームアプリ]
  ユーザーが生産側と消費側の両面で機能することにより、全ユーザーで一緒に発想力を磨いていくアプリである。自分よがりの発想力ではなく、他人にも共感できる発想力を身につけさせることが目的である。
	\begin{description}
 	\item[レビュー内容]
	ユーザー依存型アプリは投稿が増えないと開発が進まない可能性があるため、どうしたらユーザー同士で活発に活動してもらえるか考えるべきである。
	 \end{description}
	 
 \item[案4 1問1答共有アプリ]
  ユーザーが作った1問1答を共有するアプリである。問題を作る楽しさと問題を解く楽しさをシェアすることができる。
	\begin{description}
 	\item[レビュー内容]
	案3と同様に、どうしたらユーザー同士で活発に活動してもらえるか考える必要がある。投稿者が何か得をするシステムにしなければ問題の投稿数は増えていかないだろう。
	 \end{description}
	 
 \item[案5 外遊び支援アプリ]
遊びの教育を行う。IT化が進み、外で遊ぶことが少なくなってきている子供たちに対し、ITを活用することで子供に外で遊んでもらう機会を増やす。その1つの案として、GPS機能を使って鬼ごっこを行う。
	\begin{description}
 	\item[レビュー内容]
	このアプリを開発するのであれば、楽しく開発を行えるだろう。しかし開発者が楽しくてもユーザーが楽しいとは限らない。ユーザーがどのような遊びを求めているか調査する必要があるだろう。また、アプリを使いながら鬼ごっこをすると、歩きスマホのような状態になり危険なのではないか。
	 \end{description}

 \end{description}
 
 \par レビューを受け、グループ内で検討した結果、前期では「案2 プログラミングを学ぶゲーム」を作成することに決定した。
 
 \begin{figure}[H]
\begin{center}
\includegraphics[width=12cm, bb=0 0 329 162]{img/AppIdea.png}
\end{center}
\caption{絞った5種類のアプリ案}
\end{figure}
 
 \bunseki{熊谷優斗}
 
 \subsection{大学生向けプログラミング入門アプリ}
\par 前述の「案2 プログラミングを学ぶゲーム」についてより深く考えていった結果、「既存の類似アプリと相違点を持たせるため、函館要素を追加しよう」、「子供は地域性に対してあまり興味を示さない、ならば対象ユーザーを未来大生にしよう」といった理由から、未来大学に入学することが決まった高校生に対するProcessing導入アプリを作成する方針が決まった。アプリの目的は、未来大1年生が「情報表現入門」でプログラミング言語「Processing」を学ぶ際につまづきやすいポイントを、入学前にゲーム形式で気軽に学んでもらうことである。図3.5と図3.6はアプリ内画面のイメージ図の一部である。

\par 図3.6、通称「プログラミング画面」では、Scratchのように主人公の行動を表すブロックを組み立ててゆく。どのようなブロックを組めばいいかを考え、それにより主人公を操り、敵を全て倒すことがゲームの目的である。更に、「コスト」という概念を定義し、ブロックそれぞれをコストで重み付ける。1ターンで使用できるコストの上限は決まっているので、ユーザーはどのようにブロックを組めば同じコストでも最も多く行動できるかを考える必要がある。例えば、ループ文を使うことで、単純に同じブロックを何度も使用するよりも少ないコストで済む。これによりユーザーは自然と良いアルゴリズムを学ぶことができる。
\par このアプリ案をティーチングアシスタント、担当教員、企業講師の方々に見せたところ、様々なレビューを受けた。一部を抜粋すると次のようなものである。
\begin{itemize}
 \item このアプリをプレイしたところで、本当にプログラミングの教育になるのだろうか。肝心のプログラミング画面の内容が薄く、未来大生がつまづきやすいポイントを学べるとは思えない。どうすればユーザーへの「教育」になるかを練り直すべき。
 \item 大学生が使うにしては、プログラミング画面の内容が低年齢向けである。
 \item もしアプリを一般向けにリリースすることを目標としているのであれば、未来大学入学向けアプリというのは対象ユーザーが狭すぎる。
 \end{itemize}
\par このレビューを受けて、もう一度メンバー内で教育要素について考え直し、対象ユーザーを全国の高校生、大学生向けへと変更した。また、プログラミング画面にて、「→」「パンチ」といった簡単な記述ではなく、「move(right, 3)」「attack(up)」といった、より本物のソースコードに近い形で表示するようにし、そのソースコードはボタンをタップしていくことで組み立てることができる仕組みにした。

\begin{figure}[H]
\begin{center}
\includegraphics[width=12cm, bb=0 0 1182 571]{img/LengedOfN_map.png}
\end{center}
\caption{マップ画面}
\end{figure}

\begin{figure}[H]
\begin{center}
\includegraphics[width=12cm, bb=0 0 1173 563]{img/LegendOfN_programming.png}
\end{center}
\caption{プログラミング画面}
\end{figure}
 \bunseki{熊谷優斗}
 
 \subsection{中学生向けプログラミング支援アプリ}
\par プロジェクトを進めるうちに、大学生向けプログラミング入門アプリでは、プロジェクトとしての背景が、客観的に見て共感されないような内容であることに気がついた。そこで、日本の中学校ではプログラミング教育が義務化されている、また、現状のアプリ内容であれば、小学生、中学生が利用しても問題がないといった理由から、対象ユーザーを変更すべきであるという結論にたどり着いた。この日の議論により生まれたのが現状の背景(第1章にて記載)であり、現状のアプリ案(第4章にて記載)である。

\bunseki{熊谷優斗}

%4章
%%%%%%%%%%%%%%%%%%%%%%%%%%%%%%%%%%%%%%%%%%%%%%%%%%%

\chapter{前期の開発アプリについて}
\section{概要}
開発するアプリは中学生でプログラミングを習った人、興味を持った人を対象としたソースコードの組み方を学ぶゲームアプリである。図4.1のように、このゲームには自機と敵機があり、ユーザーはマス目上のステージにある自機をソースコードを組むことによって動かし、敵機を倒すことでゲームがクリアとなる。

%\\
\begin{itemize}
\item 各ステージのクリアまでの流れ
\end{itemize}
\begin{enumerate}
\item 自機を敵機の前まで移動して倒すソースコードを考える。
\item ソースコードを入力する。
\item ソースコードを実行する。
\item ソースコードの通りに自機が動く。
\item 敵を全て倒すとステージクリアとなる。
\item 使用したコストに合わせてランクとコメントが表示される。
\end{enumerate}
\par 敵機まで移動して倒すまでをいかに短いソースコードで完了させるかを目指すゲームである。

\begin{figure}[h]
\begin{center}
\includegraphics[width=12cm, bb=0 0 573 263]{img/4thParagraph/game.png}
\end{center}
\caption{ゲーム概要}
\end{figure}

\bunseki{新保遥平}

\section{プログラミング画面}
プログラミング画面ではユーザーが敵機を倒すためのソースコードを入力する。例えばfor文を入力したいときは図4.2のように画面に配置されたソースボタンをタップする。

\begin{figure}[H]
\begin{center}
\includegraphics[width=12cm, bb=0 0 564 129]{img/4thParagraph/forButton.png}
\end{center}
\caption{ソースコードの入力}
\end{figure}
ユーザーは図4.3のように画面右側に配置されたそれぞれのソースボタンをタップして、ソースコードを組み立てていく。現状、実装したソースボタンはattack()、move()、left、right、0〜9、;  などである。タップされたソースボタンは順に、右側のスペースに記述される。


\begin{figure}[H]
\begin{center}
\includegraphics[width=12cm, bb=0 0 1136 662]{img/4thParagraph/Prog-ra_programming.png}
\end{center}
\caption{プログラミング画面}
\end{figure}

例えば下記のようなプログラムを組むこととする。
\par move(right,3);
\par move(up,3);
\\
\\
このソースコードを図4.4のプログラミング画面に入力した。

\begin{figure}[H]
\begin{center}
\includegraphics[width=12cm, bb=0 0 1136 662]{img/4thParagraph/Prog-ra_programming1.png}
\end{center}
\caption{ソースコード入力後のプログラミング画面}
\end{figure}

このようにプログラムをタップで組むことが出来る。また、図4.5のようのに、次に引数である数字を入れるべきところに「up」をタップしてしまうなど、間違ったタイミングでソースボタンを押すと画面上にエラーが出て、すぐに確認ができる。


\begin{figure}[H]
\begin{center}
\includegraphics[width=12cm, bb=0 0 1024 598]{img/4thParagraph/error.png}
\end{center}
\caption{ソースコードがエラー時のプログラミング画面}
\end{figure}
ユーザーはソースコードを入力後、プログラミング画面の左上の「下矢印」ボタンで戦闘画面に移動する。

\bunseki{新保遥平}

\section{戦闘画面}
図4.6の戦闘画面はプログラミング画面で入力したソースコードで自機を動かすための画面である。戦闘画面の左下にある三角の実行ボタンを押すことで、自機を動かすことが出来る。また戦闘画面、左下の「P」と書かれたボタンでプログラミング画面に戻ることができる。現状では、あらかじめ設定されたプログラムでしか自機を動かすことが出来ない。

\begin{figure}[H]
\begin{center}
\includegraphics[width=12cm, bb=0 0 1136 662]{img/4thParagraph/Prog-ra_Battle.png}
\end{center}
\caption{戦闘画面}
\end{figure}

\bunseki{新保遥平}

\subsection{ゲーム性}
ただプログラミングを学ぶのではなく、ゲームを通してプログラミングを学ぶことでユーザーのモチベーションを保ちつつ、アプリを使ってもらえると考えた。また実際にソースコードを組むことで自機を思い通りに動かすことが出来たときにプログラミングの学習が深まると考えた。
 
\bunseki{新保遥平}


\subsection{教育性}
このアプリではユーザーがより簡潔なソースコードを組み立てられるようになるために、コストとランクがある。ソースのボタンそれぞれにコストが設けられており、問題をクリアした際にコストの使用量が少ないほど簡潔にソースコードを組み立てることが出来たと判定し、図4.7のようにランクを与える。ランクが低かった場合、より良いランクにつながるヒントを与える。そして高いランクが与えられたときに、ユーザーを褒める言葉を表示する。このサイクルが次の問題への意欲につながり、より簡潔なソースコードを組み立てることが出来るようになる。この流れをユーザーストーリーとしたものを図4.8に示す。

\begin{figure}[H]
\begin{center}
\includegraphics[width=12cm, bb=0 0 538 376]{img/4thParagraph/cost.png}
\end{center}
\caption{ランクとヒント}
\end{figure}

\begin{figure}[H]
\begin{center}
\includegraphics[width=14.5cm, bb=0 0 564 129]{img/4thParagraph/userstory.png}
\end{center}
\caption{ユーザーストーリー}
\end{figure}







\bunseki{新保遥平}
%%%%%%%%%%%%%%%%%%%%%%%%%%%%%%%%%%%%%%%%%%%%%%%%%%%

%5章
\chapter{前期の結果}
\section{プロジェクトの評価}
本プロジェクトは、多くの方からレビューを受け今の背景、課題、目的になっている。しかし、私たちが考えたアプリはその課題を解決できるアプリとなっていない。そのため、アプリの設計を見直す必要がある。

また、7月に行われた中間発表会の評価シートの結果からは、「声がはっきり聞こえた」、「声は大きく聞きやすかった」、「相手の顔を見て話してくれたので、聞き取りやすかった」などの意見をいただき、発表技術に関しては高い評価を得られた。しかし、発表内容に関しては「最終的なゴールは?」、「内容がわからないないので評価不能」、「既存のもとの比較がない」などの意見をいただいた。いただいた意見をまとめると、本プロジェクトは目標が決まっておらず、内容がわかりづらいという評価であった。

これらのことから、本プロジェクトは第三者の方に伝えられる内容になっていないので背景、課題、目的につながるアプリ案が必要である。そして、第三者の方に伝えられる内容にしていかなければならない。
\bunseki{中進吾}

\section{プロジェクトの成果}
\par 前期の活動の成果は以下の3点である。
\begin{itemize}
\item スクラッチワークショップに参加したことにより、プログラミング初心者の方にプログラミングを教える場合、C言語やJavaから始めるのではなく、Scratchのようなビジュアルプログラミング言語から始めた方が良いということがわかった。また、プログラミングで音声機器などの機械を動かしてもらうことにより、プログラミングに興味を持ってもらうことができるということがわかった。
\item 
本プロジェクトでは、Swift言語を用いてアプリ実装を行うことになっていた。しかし、メンバー全員Swift言語は使ったことがなかったため、実装に不安があった。アプリを実装できる期間は短かったが、キャラクターを動かしたり、ボタンをタップすることでソースコードを打ち込めることができるところまでアプリを開発することができた。これによって、今後のアプリの実装に対する不安がほとんどなくなった。また、アプリの実装を行えたことにより、実装に携わったメンバーはSwift言語に自信を持つことができた。

\item 中間発表会で展示するポスターは、Adobe Illustratorという描画ツールソフトを使用して作成することになっていた。このソフトを使用したことがあるのはメンバーの5人中2人だけだったため、ポスターを作成できるのは2人だけだった。しかし、3.1.5で述べたように目的と制作物がずれていたため、メンバー 5人全員で、一度 背景、目的、課題の見直しを行い、ポスターを作り直すことになった。その結果、今までAdobe Illustratorを使ったことがない人も使えるようになり、メンバー全員がポスターを作成できるようになった。11月に開催されるアカデミックリンクや最終成果発表でもポスターを使用するので、メンバー全員がポスターを作成できるようになったことは、非常に大きな成果である。
\end{itemize}

\bunseki{中進吾}


%6章
%%%%%%%%%%%%%%%%%%%%%%%%%%%%%%%%%%%%%%%%%%%%%%%%%%%

\chapter{後期の背景}

\section{プログラミング基礎の現状}

\section{現状と課題}


%7章
%%%%%%%%%%%%%%%%%%%%%%%%%%%%%%%%%%%%%%%%%%%%%%%%%%%

\chapter{後期のプロジェクトの目標}

\section{開発アプリの目標}

\section{プロジェクト学習としての目標}


%8章
%%%%%%%%%%%%%%%%%%%%%%%%%%%%%%%%%%%%%%%%%%%%%%%%%%%

\chapter{後期の活動}


\section{イベント}

\subsection{後期キックオフ会}

\subsection{アカデミックリンク}
2015年11月14日に「HAKODATEアカデミックリンク2015」という函館市内などの教育機関による
合同研究発表会が行われた。このアカデミックリンクには未来大の多くのプロジェクトが参加しており、我々のプロジェクトも参加した。教育系グループはC-mationのデモとポスターをお客さん
に見せながらポスターセッションを行った。このアカデミックリンクには様々な年齢層の方がいらっしゃり、我々のアプリについて多くの意見をいただくことができた。高校生からは「こんなアプリあったらやってみたい!」、プログラミングの経験者からは「このアプリは誰かに実際に使って評価を得ることができたの?」、「本当にC言語でつまづく部分は配列なの?」などの厳しい意見を得ることができた。このアカデミックリンクではアプリの評価だけでなく、自分たちの発表する力を試すこともできたと思われる。4月にプロジェクト学習が始まって以来、何度も発表の機会があったのでアカデミックリンクでは発表に関しては成功したと思う。
\bunseki{新保遥平}

\subsection{最終成果発表会}


\section{アプリ案の推移}

\subsection{夏休みの課題と調査報告}

\subsection{processingデバッガツール}

\subsection{C-mation}


%9章
%%%%%%%%%%%%%%%%%%%%%%%%%%%%%%%%%%%%%%%%%%%%%%%%%%%

\chapter{後期の開発アプリについて}

\section{概要}
最終成果発表に向けて開発したものは「C-mation」というC言語のプログラミング学習支援ツールである。対象者は未来大1年生である。ユーザーにはこのWebアプリを用いてC言語のプログラミングを学んでもらおうと考えた。このアプリではアニメーションを用いた概念、例題、確認問題という各段階を踏むことによってC言語を理解してもらうことを目的としている。今回は特にC言語の配列の部分の開発を行った。
\bunseki{新保遥平}

\section{コンテンツ}

\subsection{概念}
概念の部分ではまず身の回りにあるものを例に配列の説明をアニメーションを用いて行った。配列を説明するにあたり図9.1のように、変数とはなんなのか、変数を家(一戸建て)に例えて説明を行った。次に変数と配列がどこが違うのかを配列をアパートを例にして説明を行った。最後に配列を使うことによって変数より、どのような時に便利なのかを説明した。また、実際に配列の宣言の方法も説明している。
\begin{figure}[H]
\begin{center}
\includegraphics[width=10cm, bb=0 0 850 638]{img/9thParagraph/gainen_01.png}
\end{center}
\caption{概念スライド}
\end{figure}

\bunseki{新保遥平}

\subsection{例題}
例題の部分ではまず、図9.2のように例題のソースコードをユーザーに見せて、このソースコードがどのように動くのかを考えさせる。これよって、次の画面でユーザーが頭の中で考えていたソースコードの動きが合っているかを確認することができる。

\begin{figure}[H]
\begin{center}
\includegraphics[width=10cm, bb=0 0 852 638]{img/9thParagraph/reidai_01.png}
\end{center}
\caption{例題スライド1}
\end{figure}

次に図9.3、図9.4のようにアパートとソースコードを組み合わせて説明を行う。具体的には、1行目のソースコードのではapart[0]には住人が0人、apart[1]には住人が1人、apart[2]には住人が2人、apart[3]には住人が4人が未来アパートに住んでいることをアニメーションで説明している。次に2行目のソースコードのではapart[2]の表示を行うことを示している。また、ソースコードが示している部分がアパートのどのような状態かを合わせて説明している。これによって、ユーザーがソースコードの動きを見ながら配列を理解することができると考えた。

\begin{figure}[H]
\begin{center}
\includegraphics[width=10cm, bb=0 0 850 640]{img/9thParagraph/reidai_02.png}
\end{center}
\caption{例題スライド2-1}
\end{figure}

\begin{figure}[H]
\begin{center}
\includegraphics[width=10cm, bb=0 0 850 639]{img/9thParagraph/reidai_03.png}
\end{center}
\caption{例題スライド2-2}
\end{figure}

\bunseki{新保遥平}

\subsection{確認問題}
確認問題ではインタラクティブな学習を行ってもらうために、図9.5、図9.6のように、実際にユーザーに入力をして問題をといてもらうことを目的とした。
\begin{figure}[H]
\begin{center}
\includegraphics[width=10cm, bb=0 0 798 516]{img/9thParagraph/kakuninmondai_01.png}
\end{center}
\caption{確認問題1-1}
\end{figure}

\begin{figure}[H]
\begin{center}
\includegraphics[width=10cm, bb=0 0 647 422]{img/9thParagraph/kakuninmondai_02.png}
\end{center}
\caption{確認問題1-2}
\end{figure}

ユーザーには図9.7、図9.8のように確認問題のソースコードを見ながら確認問題を解いてもらう。この確認問題は未来大の1年生が「プログラミング基礎」の授業で扱う講義スライドを参考に我々が考えた問題である。問題の内容は、与えられた配列に入った8つの値の指定された大小の値を求めるときの繰り返す回数を考えさせるものである。流れとしては、ユーザーが答えだと思う値をキーボードから入力する。

\begin{figure}[H]
\begin{center}
\includegraphics[width=10cm, bb=0 0 557 576]{img/9thParagraph/kakuninmondai_03.png}
\end{center}
\caption{確認問題1-3}
\end{figure}

\begin{figure}[H]
\begin{center}
\includegraphics[width=10cm, bb=0 0 557 521]{img/9thParagraph/kakuninmondai_04.png}
\end{center}
\caption{確認問題1-4}
\end{figure}

すると、図9.9、図9.10のように入力した値が正しいか、間違いかをすぐ表示してくれる。そして入力した値が間違っている場合、その後プログラムがどうなるかを説明してくれる。これは、ユーザーがなぜ「この値が間違っているのか」を理解してもらうためである。このようにして、確認問題でユーザーが本当に配列を理解したのかを確かめている。


\begin{figure}[H]
\begin{center}
\includegraphics[width=10cm, bb=0 0 645 418]{img/9thParagraph/kakuninmondai_05.png}
\end{center}
\caption{確認問題1-5}
\end{figure}

\begin{figure}[H]
\begin{center}
\includegraphics[width=10cm, bb=0 0 644 419]{img/9thParagraph/kakuninmondai_06.png}
\end{center}
\caption{確認問題1-6}
\end{figure}

\bunseki{新保遥平}

%10章
%%%%%%%%%%%%%%%%%%%%%%%%%%%%%%%%%%%%%%%%%%%%%%%%%%%

\chapter{後期の結果}

\section{プロジェクトの評価}

\section{プロジェクトの成果}


%11章
%%%%%%%%%%%%%%%%%%%%%%%%%%%%%%%%%%%%%%%%%%%%%%%%%%%

\chapter{今後の課題と展望}
\section{アプリ案の課題と展望}
中間発表会で得られたレビューをまとめると、現状のアプリ案は、目標がしっかりと定まっておらず、内容がわかりにくい。今後は、現在のアプリ案を再考し、より具体的で背景、課題、目的が一貫性のあるアプリ案にしていく必要がある。そこで現状のアプリ案を細かく修正していくのか、それとも、また1からアプリ案を考え直し、要件定義をやり直すかは、これからグループで話し合って決めるべき課題である。
\par
話し合いで決めたアプリ案を教育アプリとして体裁を整え、11月に開催されるアカデミックリンクまでに実際に動かすことが出来るプロトタイプを完成させる。アカデミックリンクでワークショップを開き、参加者からプロトタイプのレビューを受け、そこで得たレビューを活かしてアプリの改善を行うことが今後の展望である。
\bunseki{皀勢也}

\section{メンバーの課題と展望}
中間発表会に向けて、プロトタイプを制作する実装班と、ポスターを制作するポスター班で分かれたが、いくつか課題があった。アプリやポスターに使われる画像の制作は情報デザインコースのメンバーがほとんど一人で作っていたため、実装がスムーズにいかなかった。なるべくタスクをメンバーで分散させるべきであった。また、ポスター班は実装に関わっていなかったため、実装班とのスキルの差がある。後期からは全員で実装を行っていくため、ポスター班はSwift言語とバージョン管理システムに慣れていく必要がある。
\par
さらに、プロジェクト学習の授業外での作業が多く、メンバー全員が作業できない日があった。そのため、情報共有に時間がかかったり、効率良く作業をすることができなかった。
\par
今後の展望は、後期からはメンバー全員で実装し、スムーズにアプリ開発を行えるように、タスクを分散させてプロジェクトを進めていき、グループ全体の作業は授業時間内に終わらせることである。また、経験したことない作業でも積極的に取り組み、1人のメンバーにタスクが集中しないように協力していくことである。

\bunseki{皀勢也}


%12章
%%%%%%%%%%%%%%%%%%%%%%%%%%%%%%%%%%%%%%%%%%%%%%%%%%%

\chapter{学び}
\section{グループとしての学び}
アプリ案の要件定義を固めずに実装を行ったため、対象ユーザや目的が合っていないことが分かった。また、教育系グループとして教育という意味をしっかり決めていなかったため、プロジェクトの目的を見失い要件定義を一からやり直すことになった。そのため、時間をかけて要件定義を行うことの重要性を学んだ。また、プロジェクト学習の授業外での作業が多く、議事録を残していないことがあり、情報共有がうまくできていなかった。このことからドキュメントを残して、情報共有することの大切さを学んだ。
\par
グループでの活動を通してグループメンバーの得意分野、不得意分野を把握し、各メンバーがそれぞれの得意分野を活かせる作業を行うことができた。また、話し合いを重ねることでスムーズに進めることができ、メンバーの適切な役割分担を学んだ。
\bunseki{皀勢也}

\section{各メンバーの学び}
\subsection{熊谷優斗}
\par これまで大学での活動の中で何度かPBLは経験したものの、自ら課題を考え主体的に活動したことは初めての経験だったため、様々な学びを得ることができた。その中でも、今回は2つの学びをとりあげる。1つ目の学びは、メンバーに技術を教えることの難しさだ。過去のPBL経験から、Gitに関しての知識は持っていたため、メンバーにGitとGitHubの使い方を教える役目を受け持った。しかし、メンバーによって理解度の差があるにも関わらず、理解できているメンバーに合わせて話を進めていってしまった。そのため、メンバー全員にGitの使い方を理解してもらうことができなかった。このことから、メンバーの理解状況を確認し、全員が理解した上で次のステップに進む、という教え方が大切であることを学んだ。2つ目の学びは、適切な文言を使用することの重要性だ。前期の活動では、「基本的な制御文の書き方を学ぶ」という意味で、「簡単なアルゴリズムを学ぶ」という文言を使用していた。その結果、TAや教員によるレビューの際に、自分たちが作成したいアプリケーションの内容をうまく伝えることができず、苦労した。このことから、適切な文言を使用することは、他人に自分の意見を伝えるために重要であることを学んだ。
\bunseki{熊谷優斗}

\subsection{皀勢也}
本プロジェクトを進めるにあたって、Swift言語を使ってプログラミングを行うため、Xcodeが入っているMacOSが必要であった。自分はWindowsのノートパソコンしか持っていなかったため、enPiTからMacBook Airを借りて作業を行った。初めは、Windowsと異なり上手く作業を進めれなかったが、使っているうちにMacOSに慣れて上手く作業できるようになった。
\par
また、グループで開発を行っていくうちに、他者にもわかりやすいソースコードを書くように意識した。さらに、GitとGitHubを使ってバージョン管理することや議事録を残すことの重要性を学んだ。

\bunseki{皀勢也}

\subsection{新保遥平}
\par この1年間では多くのことを学んだ。技術的な点では主に3つの学びを得た。1つめにGitHubをしっかりと理解して使えるようになったことだ。GitHubは以前に使ったことがあったが、曖昧な部分が多くわからない部分も多くあった。だが、プロジェクトが始まってからはグループ内でもGitHubを上手く使ってアプリの開発や報告書の作成を行うことができた。2つ目に、ポスターの作成のためにAdobe IIIustratorの使い方を学ぶことができた。他プロジェクトではポスターの作成はデザインコースの人が作るプロジェクトが多い中、自分もポスター作成に携わることができ、いい経験になった。3つ目は発表技術である。私は即席で人の前で話すことがとても苦手であったが、この一年、何度も即席で話す機会が多くあり、自分の意見をまとめて話すことができるようになった。
\par 

\par 前期のプロジェクトリーダーとしての学びは、人に仕事を振リ分けることの重要差だった。前期は自分一人で行うべきでないタスクも一人で行ってしまったため、自分のタスクの進捗が遅れることがあった。

\par 後期のプロジェクトリーダーとしての学びは情報の共有だった。後期からはプロジェクトが始まる前に今日の活動のアジェンダ、終わる際に、今日の進捗報告を行った。これは


\bunseki{新保遥平}

\subsection{中進吾}
\par 仕事を均等に振り分けることができていなかったため、進捗が遅れることがあった。グループリーダーとして後期は、できるだけ仕事が偏らないように心掛け、プロジェクトをまとめていきたい。
\par また、私はこれまでに2度PBLに参加してきたが要件定義は先輩に任して、自分は実装ばかりをやっていた。今回、一から要件定義を考えたことは大きな学びであり、自分に足りないものを見つけることができた。今後、PBLに参加するときには自分から率先して要件定義を立て、この経験を後輩に伝えていきたい。
\bunseki{中進吾}

\subsection{矢吹渓悟}
デザインプロセスの大切さを再認識し、同時に自分の未熟さを痛感した。なぜなら、初期テーマの設定からブレインストーミングやフィールドの調査を怠ってしまったからだ。本来ならば、メンバーが一丸となって、話し合いや現状を調べながら、みんなでテーマを確立するべきだ。しかしながら、今回は個人個人がやりたいものを考え、そこから一つに絞ってしまったのだ。そのため、背景情報や対象ユーザーの設定が偏見や想像論になったり、後付けとなり、プロジェクトの根幹が揺らいでしまい、最終的にテーマの見直しまでに落ちてしまった。よって、今後テーマを見直す上で大切なのは、ブレインストーミングなどを通していろいろな可能性を吟味した上で、徐々に一つに絞り込むことが大切だと再認識した。
\bunseki{矢吹渓悟}




% 以降、付録(付属資料)であることを示す
\begin{appendix}

\chapter{新規習得技術}
\par【Swift】
\par《キーワード》
\par Swift・Xcode・SpriteKit・iPhone・iPad・Mac・オブジェクト指向・iOSアプリケーション
\par《概要》
\begin{itemize}
\item 2014年にWWDCで発表された、新しいオブジェクト指向型のプログラミング言語
\item Objective-Cに代わるiOSアプリケーション開発言語
\item 開発環境はMac、実機テスト用にiPhoneやiPadが必要で、開発エディタはXcodeが推奨される
\item 開発する上で、iOS Developer Programへの登録が必要
\item SpriteKitという2Dカジュアルゲーム用のフレームワークなど、幾つかテンプレートが用意してある
\end{itemize}
\par《Swiftの特長》
\begin{itemize}
\item 速い:Objective-Cより実行速度が高速
\item モダン:プログラミングの書き方が新しい
\item 安全:プログラミングにエラーが起きにくい仕組みが増える
\end{itemize}
\par《Objective-Cとの主な相違点》
\begin{itemize}
\item 行末のセミコロンや制御文の()が不要
\item メソッドの記述方法が「.メソッド名()」と一般的な記述に
\item 変数にnilが代入されるとエラーが表示され、安全性が向上した
\item 別クラスにアクセスするときもimportが不要
\end{itemize}
\par《Xcodeとは》
\begin{itemize}
\item iPhoneアプリを作るための開発ツール
\item アプリを作るのに必要な作業を全て行う
\end{itemize}
\par(ex:アプリ画面のデザイン・プログラムの入力・実行ファイルの作成)
\begin{itemize}
\item iOSシュミレーターで実機を使わなくても粗方のデモストレーションが行える
\end{itemize}
\par《SpriteKitを使うメリット》
\begin{itemize}
\item 文字やグラフィックスを素早く表示させたり動かしたりすることができる
\item 2Dの物理エンジンがついているため、物理的な動きをシュミレートさせることができる
\end{itemize}

\par【Git / GitHub】
\par《キーワード》
\par Git・GitHub・分散型・バージョン環境システム・リポジトリ・ローカル・保存(コミット)・オープンソース
\par《概要》
\par[Git]
\begin{itemize}
\item プログラムソースなどの変更履歴を管理する、分散型のバージョン管理システムのこと
\item Linuxの開発チームが使用していたことがきっかけとなり、徐々に世界中に広まった
\end{itemize}
\par[GitHub]
\begin{itemize}
\item Gitの仕組みを利用して、世界中の人々が自分の作品(プログラミングコードやデザインデータ、ドキュメントなど)を保存・公開することができるウェブサービスのこと
\item 運営はGitHub社で、個人・企業を問わず無料で行うことができる
\item 基本的にオープンソースだが、有料サービスを利用するとプライベートなリポジトリも作ることができる
\end{itemize}
\par《Gitと従来品の比較》
\par[Git]
\begin{itemize}
\item 自分のパソコンなどのローカル環境に、全ての変更履歴を含むリポジトリが複製される
\item 各ローカル環境がリポジトリのサーバーとなれる
\item ローカル環境にもコードの変更履歴を保存(コミット)できるので、リモートのサーバーに接続する必要がない
\item ネットワークに接続していなくても作業ができる
\end{itemize}
\par[従来]
\begin{itemize}
\item サーバー上にある一つのリポジトリを共同で使っていた
\item 利用者が増えると変更内容の衝突が頻繁に起きる
\item 整合性を維持するのが大変
\end{itemize}


\chapter{活用した講義}
%\begin{hissu}
\par【情報デザイン1】
\par《キーワード》
\par Adobe Illustrator・図解表現・ポートフォリオ
\par《授業内容》
\begin{itemize}
\item Adobe Illustratorの使い方(個人)
\item 市立函館博物館を題材とした図解表現(個人)
\item 図解表現の説明も兼ねたポートフォリオの作成
\end{itemize}
\par《活かせる技術・知識・経験》
\begin{itemize}
\item 画像やアプリ素材をAdobe Illustratorで作成すること。
\item 議事録や発表ポスターの素材を図解を用いて表現すること。
\end{itemize}

\par【情報デザイン2】
\par《キーワード》
\par グループワーク・タンジブル・アクティングアウト(寸劇)・プロトタイピング・ストーリーテーリング・プレゼンテーション・ポートフォリオ・図解
\par《授業内容》
\begin{itemize}
\item 時計の分析(グループ)
\item スマートウォッチの分析(グループ)
\item タンジブルな提案(グループ)
\end{itemize}
\par《活かせる技術・知識・経験》
\begin{itemize}
\item 机に座って話し合いするよりも、手足を動かして考える方がより良い提案になること。
\end{itemize}

\par【情報表現基礎3】
\par《キーワード》
\par グループワーク・観察・フィールドワーク・アクティングアウト(寸劇)・プロトタイピング・ストーリーテーリング・スケージューリング・プレゼンテーション・ポートフォリオ・図解
\par《授業内容》
\begin{itemize}
\item カバンのスケッチを通した観察(個人)
\item オートマタ制作(個人)
\item 西部地区のフィールドワークを通して、より西部地区に足を運べる物・事・企画の提案(グループ)
\item 各ポートフォリオの作成
\end{itemize}
\par《活かせる技術・知識・経験》
\begin{itemize}
\item 情報共有をしっかりと行わないと、メンバー間の考えのズレやタスクの進捗に影響すること。
\item 提案のユーザーストーリーを考えて、提案がユーザーにどういう変化を与えるのかを常に考えながら、作成していくこと。
\item 提案をプロトタイピングし、デモストレーションやアクティングアウトなどを行う。それを通して、問題点・優良点・改善点などを発見しやすくすること。
\item プレゼンテーションにアクティングアウト(寸劇)や提案のデモストレーションを盛り込んで、発表に説得力を持たせること。
\end{itemize} 


%\chapter{相互評価}
%\begin{hissu}
%課題解決過程で分担し、連携した作業全般について、互いに客観的に評価する。 
%\end{hissu}

%\chapter{その他製作物}
%\begin{hissu}
%その他成果物をプロジェクトの担当教員の指示に従って添付する。
%\end{hissu}

%付録の終わり
\end{appendix}


%\backmatter

% 参考文献
\begin{thebibliography}{9}
% \bibitem {ラベル} 著者名. 書籍名. 出版社,  年号.
% \bibitem {A2} ほげほげお. うんたらかんたら,  2003.
 \bibitem {A2} Code部. 5歳からプログラミング必修化!?世界の最新IT教育トレンドまとめ |Code部,  2015. \url{http://blog.codecamp.jp/programming_education/} (2015/7/20)
 \bibitem {A2} TechAcademy. プログラミングが義務教育に!政府の成長戦略素案に盛り込まれたプログラミング教育の内容とは |TechAcademyマガジン,  2013. \url{http://techacademy.jp/magazine/736} (2015/7/20)
  \bibitem {A2} Code部. 大人も子供も楽しめる!プログラミング入門ゲーム「Scratch」をやってみた|Code部 ,  2015. \url{http://blog.codecamp.jp/try_scratch/} (2015/7/20)
 \bibitem {A2} 日本経済新聞. 中学の技術・家庭科で「ビジュアルプログラミング」を導入:日本経済新聞,  2012. \url{http://www.nikkei.com/article/DGXNASFK2701H_X21C12A2000000/} (2015/7/20)
 \bibitem {A2} ヴィストン株式会社. 計測器プログラマー |ヴィストン株式会社,  2012. \url{http://www.vstone.co.jp/products/mcprogrammer/} (2015/7/20)
  \bibitem {A2} コードアカデミー高等学校. コードアカデミー高等学校,  2015. \url{http://www.code.ac.jp/} (2015/7/20)
\end{thebibliography}

\end{document}